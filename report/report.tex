\documentclass{article}
\title{Programming Language Processor \\ Report}
\author{slackhideo}
\date{\today}

\def\reporttrue{\let\ifreport=\iftrue}
\def\reportfalse{\let\ifreport=\iffalse}
%\reportfalse  % question
\reporttrue  % report
\usepackage{rail}


\begin{document}
\ifreport
\maketitle
\else
\begin{center}
{\huge Programming Language Processor Assignment 3}
\end{center}
\fi


%%%%%%%%%%%%%%%%%%%%%%%%%%%%%%%%%%%%%%%%%%%%%%%%%%%%%%%%%%%%%%%%%%%%%%%%%%%%%%%%%%%%%%%%%%%%%%%
\section*{Question 1}

To introduce the following do-while statement to PL/0', answer the following questions.
\begin{description}
 \item[Production rule] {\it statement}  $\to$ {\bf do} {\it statement} {\bf while} {\it condition}
 \item[Action] A statement '{\bf do} {\it statement} {\bf while} {\it condition}' works as follows
	    \begin{enumerate}
	     \item Execute {\it statement}.
	     \item If the value of {\it condition} is true, go to the step 1. Otherwise, exit this loop.
	    \end{enumerate}
\end{description}

\subsection*{Question 1-1}
To add a token {\tt do} to a set of starting tokens of {\it statement},
modify a function {\tt isStBeginKey} in {\tt compile.c} and explain the modification in your report.

\ifreport
(Answer)\\
\fi
% Write your answer.

We add the {\tt case Do:} part to the {\tt isStBeginKey} function because the
statement we are adding begin with the {\tt do} keyword.


\subsection*{Question 1-2}
Modify a function {\tt statement} in {\tt compile.c} 
so that your PL/0' compiler can output object codes of Fig.\ref{fig:dowhile-code} for
do-while statements.
Explain the modificaiton in your report.

\clearpage

\begin{figure}[h]
\begin{tabular}{ll}
label1: & Object codes of {\it statement} \\
        & Object codes of {\it condition} \\
        & jpc label2 \\
        & jmp label1 \\
label2: & \\
\end{tabular}
\caption{Object codes for a do-while statement}
\label{fig:dowhile-code}
\end{figure}

\ifreport
(Answer)\\
\fi
% Write your answer

We include the following code to {\tt statement} function:

\begin{verbatim}
case Do:                            /* do-while statement */
    token = nextToken();            /* gets the next token */
    backP2 = nextCode();            /* target address for the jump at the end */
    statement();                    /* a statement */
    token = checkGet(token, While); /* next token must be "while" */
    condition();                    /* a condition */
    backP = genCodeV(jpc, 0);       /* a conditional jump to the end */
    genCodeV(jmp, backP2);          /* a jump to the beginning of do-while
                                                                    statement */
    backPatch(backP);               /* adjusts the jpc target address */
    return;
\end{verbatim}


\subsection*{Question 1-3}
What does your PL/0' compiler outputs when your PL/0' compiler compiles
and executes a PL/0' program do.pl0 of Fig. \ref{fig:do-while}?

\begin{figure}[h]
\begin{verbatim}
var x;
begin
   x := 0;
   do begin
      write x;
      writeln;
      x := x + 1
   end
   while x < 3
end.
\end{verbatim}
\caption{A test program {\tt do.pl0}}\label{fig:do-while}
\end{figure}


\ifreport
(Answer)\\
\fi
% Write your answer.

It outputs:
\begin{verbatim}
; start compilation
; start execution
0
1
2
\end{verbatim}


%%%%%%%%%%%%%%%%%%%%%%%%%%%%%%%%%%%%%%%%%%%%%%%%%%%%%%%%%%%%%%%%%%%%%%%%%%%%%%%%%%%%%%%%%%%%%%%
\newpage
\section*{Question 2}
Answer the following questions to add the following repeat-until statement to PL/0'.

\begin{description}
 \item[Production rule] {\it statement}  $\to$ {\bf repeat} {\it statement} {\bf until} {\it condition}
 \item[Action] A statement '{\bf repeat} {\it statement} {\bf until} {\it condition}' works as follows.
	    \begin{enumerate}
	     \item Execute {\it statement}.
	     \item If the value of {\it condition} is false, go to the step 1. Otherwise, exit this loop.
	    \end{enumerate}
\end{description}

\subsection*{Question 2-1}
Write object codes for the repeat-until statement 
like object codes for the do-while statement of Fig.\ref{fig:dowhile-code}.

\ifreport
(Answer)\\
\fi
% Write your answer

\begin{figure}[h]
\begin{tabular}{ll}
label1: & Object code of {\it statement} \\
        & Object code of {\it condition} \\
        & jpc label1 \\
\end{tabular}
\end{figure}


\subsection*{Question 2-2}
Modify {\tt getSource.h} and {\tt getSource.c} to register two tokens
{\tt repeat} and {\tt until} to your PL/0' compiler.
Explain the modification in your report.

\ifreport
(Answer)\\
\fi
% Write your answer.

In {\tt getSource.h} we add the following line;

\begin{verbatim}
Repeat, Until,
\end{verbatim}

in the {\tt typedef  enum  keys KeyId} block. And in {\tt getSource.c}, we add:

\begin{verbatim}
{"repeat", Repeat},
{"until", Until},
\end{verbatim}

to add {\tt repeat} and {\tt until} as reserved words and make the compiler
recognize them.


\subsection*{Question 2-3}
To add a token {\tt repeat} to a set of starting tokens of {\it statement},
modify a function {\tt isStBeginKey} in {\tt compile.c} and explain the modification in you report.

\ifreport
(Answer)\\
\fi
% Write your answer.

We add the {\tt case Repeat:} part to the {\tt isStBeginKey} function because
 the statement we are adding begin with the {\tt Repeat} keyword.


\subsection*{Question 2-4}
Modify a function {\tt statement} in {\tt compile.c}
so that your PL/0' compiler can output object codes for repeat-until statements.
Explain the modificaiton in your report.

\ifreport
(Answer)\\
\fi
% Write your answer.

We include the following code to {\tt statement} function:

\begin{verbatim}
case Repeat:                        /* repeat-until statement */
    token = nextToken();            /* gets the next token */
    backP = nextCode();             /* target address for the jump at the end */
    statement();                    /* a statement */
    token = checkGet(token, Until); /* next token must be "until" */
    condition();                    /* a condition */
    genCodeV(jpc, backP);           /* a conditional jump to the beginning of
                                                       repeat-until statement */
    return;
\end{verbatim}


\subsection*{Question 2-5}
What does your PL/0' compiler outputs when your PL/0' compiler compiles
and executes a PL/0' program {\tt repeat.pl0} of Fig.\ref{fig:repeat-until}?

\begin{figure}[h]
\begin{verbatim}
var x;
begin
   x := 0;
   repeat begin
      write x; 
      writeln;
      x := x + 1
   end
   until x=3
end.
\end{verbatim}
\caption{A test program {\tt repeat.pl0}}\label{fig:repeat-until}
\end{figure}


\ifreport
(Answer)\\
\fi
% Write your answer.

It outputs:
\begin{verbatim}
; start compilation
; start execution
0
1
2
\end{verbatim}



%%%%%%%%%%%%%%%%%%%%%%%%%%%%%%%%%%%%%%%%%%%%%%%%%%%%%%%%%%%%%%%%%%%%%%%%%%%%%%%%%%%%%%%%%%%%%%%
\newpage
\section*{Question 3}
Answer the following questions to add the following if-then-else statement to PL/0'.

\begin{description}
 \item[Production rule] {\it statement}  $\to$ {\bf if} {\it condition} {\bf then} 
       ${\it statement}_1$ ({\bf else} ${\it statement}_2$ $\vert$ {$\epsilon$})
 \item[Action] A statement '{\bf if} {\it condition} {\bf then} 
	    ${\it statement}_1$ ({\bf else} ${\it statement}_2$ $\vert$ {$\epsilon$})'
	    works as follows.
	    \begin{enumerate}
	     \item Evaluate {\it condition}.
	     \item If the value of {\it condition} is true, execute ${\it statement}_1$.
	     \item If the value of {\it condition} is false and ${\it statement}_2$ exists, 
		   execute ${\it statement}_2$.
	    \end{enumerate}
 \item[Description] To resolve ambiguity of the grammar of PL/0', 
	    we use the following rule.
	    \begin{itemize}
	     \item When we find an {\bf else}, we relate the {\bf else} to the nearest {\bf then}
		   which has not be related to any {\bf else} yet.
	    \end{itemize}
\end{description}


\subsection*{Question 3-1}
Write object codes for 
a statement '{\bf if} {\it condition} {\bf then}
${\it statement}_1$ {\bf else} ${\it statement}_2$'
like object codes for a do-while statement of Fig.\ref{fig:dowhile-code}.

\ifreport
(Answer)\\
\fi
% Write your answer.

\begin{figure}[h]
\begin{tabular}{ll}
        & Object code of {\it condition} \\
        & jpc label1 \\
        & Object code of {\it statement1} \\
        & jmp label2 \\
label1: & Object code of {\it statement2} \\
label2: & \\
\end{tabular}
\end{figure}


\subsection*{Question 3-2}
Modify {\tt getSource.h} and {\tt getSource.c} to register a token
{\tt else} to your PL/0' compiler.
Explain the modification in your report.

\ifreport
(Answer)\\
\fi
% Write your answer.

In {\tt getSource.h} we add the following line;

\begin{verbatim}
Else,
\end{verbatim}

in the {\tt typedef  enum  keys KeyId} block. And in {\tt getSource.c}, we add:

\begin{verbatim}
{"else", Else},
\end{verbatim}

to add {\tt else} as reserved words and make the compiler recognize them.


\subsection*{Question 3-3}
Modify a function {\tt statement} in {\tt compile.c}
so that your PL/0' compiler can output object codes for if-then-else statements.
Explain the modificaiton in your report.

\ifreport
(Answer)\\
\fi
% Write your answer.

We include some code in the {\tt case If} part of {\tt statement} function, as
follows:

\begin{verbatim}
case If:                           /* if-then-else statement */
    token = nextToken();           /* gets the next token */
    condition();                   /* a conditional expression */
    token = checkGet(token, Then); /* next token must be "then" */
    backP = genCodeV(jpc, 0);      /* a conditional jump (to the end if it is
                                      an if-then statement, or to the second
                                      statement if it is an if-then-else
                                                                    statement */
    statement();                   /* a statement just after "then" */
    if(token.kind == Else) {       /* verifies if it is an if-then-else
                                                                    statement */
        token = nextToken();       /* gets the next token */
        backP2 = genCodeV(jmp, 0); /* a jump to the end of the statement */
        backPatch(backP);          /* adjusts the jpc target address */
        statement();               /* a statement after "else" */
        backPatch(backP2);         /* adjusts the jmp target address */
    }
    else {
        backPatch(backP);          /* adjusts the jpc target address */
    }   
    return;
\end{verbatim}


\subsection*{Question 3-4}
What does your PL/0' compiler outputs when your PL/0' compiler compiles
and executes a PL/0' program else.pl0 of Fig.\ref{fig:if-then-else}?

\clearpage

\begin{figure}[h]
\begin{verbatim}
var x;
begin
   x := 0;
   while x<3 do begin
      if x < 1 then write 0
      else if x < 2 then write 1
      else write 2;
      writeln;
      x := x+1;
   end;
end.
\end{verbatim}
\caption{A test program else.pl0}\label{fig:if-then-else}
\end{figure}


\ifreport
(Answer)\\
\fi
% Write your answer.

It outputs:
\begin{verbatim}
; start compilation
; start execution
0 
1 
2 
\end{verbatim}


%%%%%%%%%%%%%%%%%%%%%%%%%%%%%%%%%%%%%%%%%%%%%%%%%%%%%%%%%%%%%%%%%%%%%%%%%%%%%%%%%%%%%%%%%%%%%%%
\newpage
\section*{Question 4}
Answer the following questions to introduce one-dimensional array to PL/0'.


\subsection*{Question 4-1}
Explain how to modify the grammar of PL/0' to introduce one-dimensional array to PL/0'.

\ifreport
(Answer)\\
\fi
% Write your answer.

In order to add one-dimensional array feature in PL/0', we can do the following
changes:

\begin{rail}
    varDecl: "var" ("ident" ("[" "number" "]") ? + ",") ";"
\end{rail}

\begin{rail}
    statement: ("ident" ("[" expression "]")? ":=" expression)?
\end{rail}

\begin{rail}
    factor: "ident" ("[" expression "]")?
\end{rail}

The {\it varDecl} modification is needed to declare arrays. It accepts an
{\tt ident} as an identifier and an {\tt number} as the size.\\
The {\it statement} modification lets the array (an element of the array) to
receive a value.\\
The {\it factor} modification is for getting a value from the array (from a
specified position of the array).\\
Please note that are only shown the modified parts.
The rest of {\it statement} diagram, {\it factor} diagram and the other needed
diagrams are identical to the original ones.\\

\subsection*{Question 4-2}
Do you need new instructions to the PL/0' virtual machine for one-dimensional array?
If you need new instructions, define thier mnemonics and their actions. 

\ifreport
(Answer)\\
\fi
% Write your answer.

Yes, I needed to add two more instructions to the PL/0' virtual machine.
They are:\\

\noindent\textbf{\Large lot}\\

\noindent\textbf{Overview}\\
\indent Push a value of a variable (especially useful for arrays)
 to the stack considering the value of stack top.\\

\noindent\textbf{Mnemonic}\\
\indent \texttt{lot,LEVEL,ADDR}\\

\noindent\textbf{Description}\\
\indent LEVEL is a nesting level in a source PL/0' program.
 ADDR is a relative address.\\

\noindent\textbf{Details}\\
\indent \texttt{top--;\\
\indent stack[top] = stack[display[LEVEL] + ADDR + stack[top]];\\
\indent top++;}\\

\vspace{0.5cm}

\noindent\textbf{\Large stt}\\

\noindent\textbf{Overview}\\
\indent Store a value on the stack top to a variable on the stack,
considering the value below of stack top.\\

\noindent\textbf{Mnemonic}\\
\indent \texttt{stt,LEVEL,ADDR}\\

\noindent\textbf{Description}\\
\indent LEVEL is a nesting level in a source PL/0' program.
 ADDR is a relative address.\\

\noindent\textbf{Details}\\
\indent \texttt{stack[display[LEVEL] + ADDR + stack[top - 2]] = stack[--top];}\\


\subsection*{Question 4-3}
Modify your PL/0' compiler so that it can support one-dimensional array.
Explain the modification in your report.

\ifreport
(Answer)\\
\fi
% Write your answer.

In addition to the new instructions added in {\tt codegen.h} and
 {\tt codegen.c}, I also modified the {\tt compile.c} file:

\begin{itemize}
    \item In {\tt varDecl} function, I introduced the array declaration method
    described in Question 4-1, getting the array size as a number (I tried to
    use {\tt expression}, but determining the starting addresses of arrays
    dynamically is quite complex). To store the array in the name table, I
    use a function {\tt enterTarray}, which is similar to {\tt enterTvar}
    but uses the array size as a parameter.
    \item In {\tt statement} function, I introduced the value assignment
    method, as described in Question 4-1. Here I use the {\tt stt} instruction.
    \item In {\tt factor} function, I introduced the value retrieval method,
    as described in Question 4-1. Here I use the {\tt lot} instruction.
\end{itemize}

In {\tt getSource.h} file, I registered {\tt Lbrack} and {\tt Rbrack} in the
{\tt typedef  enum  keys KeyId}.\\

In {\tt getSource.c} file, I added the following lines to
{\tt struct keyWd KeyWdT[]}:

\begin{verbatim}
{"[", Lbrack}, /* added left square brackets (used in array declaration) */
{"]", Rbrack}, /* added right square brackets (used in array declaration) */
\end{verbatim}

and the following line to {\tt initCharClassT} function:

\begin{verbatim}
charClassT['['] = Lbrack; charClassT[']'] = Rbrack;
\end{verbatim}

Also, I added support for printing the arrays in HTML (although they are shown
in HTML files as {\tt varId}, I included a kind {\tt arrayId}, used in
internals).\\

In {\tt table.h} file, I included the kind {\tt arrayId}.\\

In {\tt table.c} file, I included the function {\tt enterTarray}, which
registers an array into the name table.\\

\subsection*{Question 4-4}
Write a simple test program {\tt array.pl0} for one-dimensional array.
Explain the test program and what your PL/0' compiler outputs when it
compiles and executes the test program.

\ifreport
(Answer)\\
\fi
% Write your answer.

A simple test program is:

\begin{verbatim}
var small[2], another[3];
begin
    small[0] := 5 + 7;
    small[1] := small[0];
    another[4 / 2] := 2 * small[1];
    write small[1];
    writeln;
    write another[1 + 1];
    writeln
end.
\end{verbatim}

This test program exercises creation of multiple arrays, assignment of
expressions of numbers, expressions using arrays and expressions to calculate
the index.\\
My PL/0' compiler outputs:
\begin{verbatim}
; start compilation
; start execution
12 
24
\end{verbatim}

%%%%%%%%%%%%%%%%%%%%%%%%%%%%%%%%%%%%%%%%%%%%%%%%%%%%%%%%%%%%%%%%%%%%%%%%%%%%%%%%%%%%%%%%%%%%%%%
\newpage
\section*{Question 5}
Answer the following questions to introduce procedures (functions withaout any return values) to PL/0'.

We use the following statement to call a procedure with $n$ arguments.
\begin{quote}
 {\bf call} {\it procedure}($arg_1$, $arg_2$, $\dots$, $arg_n$)
\end{quote}


\subsection*{Question 5-1}
Explain how to modify the grammar of PL/0' to introduce procedures to PL/0'.

\ifreport
(Answer)\\
\fi
% Write your answer.

In order to add procedure declarations and procedure calls features in PL/0',
we can do the following changes:\\

\begin{rail}
    block: (((() + constDecl)
             (() + varDecl)
             (() + funcDecl)
             (() + procDecl)) +
    ()) statement
\end{rail}

\begin{rail}
    procDecl: "procedure" "ident" "(" ("ident" + ",")? ")" block ";"
\end{rail}

\begin{rail}
    statement: (("call" "ident" "(" (expression + ",")? ")")
             | ("return" expression?))?
\end{rail}

The \emph{block} modification is needed to declare procedures. It adds a
{\tt procDecl} (procedure declaration) section.\\
The \emph{procDecl} is similar to \emph{funcDecl}, but it uses the
{\tt procedure} keyword.\\
The \emph{statement} introduces the {\tt call} statement, which accepts the
{\tt call} keyword, then an {\tt ident} as the procedure identifier and then
the parameters, if there is any. It makes possible to call procedures.
Also, {\tt return} statement was modified to accept an {\tt expression} or
not.\\
Please note that are only shown the modified parts.
The other sections are the same as the original diagram.\\

\subsection*{Question 5-2}
Do you need new instructions to the PL/0' virtual machine for procedures?
If you need new instructions, define thier mnemonics and their actions. 

\ifreport
(Answer)\\
\fi
% Write your answer.

No, I did not need any new instructions.

\subsection*{Question 5-3}
Modify your PL/0' compiler so that it can support procedures.
Explain the modification in your report.

\ifreport
(Answer)\\
\fi
% Write your answer.

I did the following modifications:

\begin{itemize}
    \item In {\tt getSource.h} file, I added {\tt Proc} and {\tt Call} in the
        {\tt typedef  enum  keys KeyId}.
    \item In {\tt getSource.c} file, I added {\tt \{"procedure", Proc\}} and
        {\tt \{"call", Call\}} entries in the {\tt struct keyWd KeyWdT[]}.
        Also, I added support for printing the procedures in HTML.
    \item In {\tt table.h} file, I included the kind {\tt procId}.
    \item In {\tt table.c} file, I included the function {\tt enterTproc},
        which registers a procedure into the name table.
    \item In {\tt compile.c} file, I did the following changes:
        \begin{itemize}
            \item In {\tt block} function, I included the {\tt procDecl()}
                function call, as shown in the \emph{block} syntax diagram of
                Question 5-1.
            \item Added the {\tt procDecl()} function, which compiles procedure
                declarations, as shown in the \emph{procDecl} syntax diagram
                of Question 5-1.
            \item In {\tt statement} function, I modified the {\tt return}
                statement to accept an {\tt expression} or no return value
                (used in the case of procedures). Also, I introduced the {\tt
                call} statement, as shown in the \emph{statement} syntax
                diagram of Question 5-1.
        \end{itemize}
\end{itemize}

\clearpage

\subsection*{Question 5-4}
Write a simple test program {\tt proc.pl0} for procedures.
Explain the test program and what your PL/0' compiler outputs when it
compiles and executes the test program.

\ifreport
(Answer)\\
\fi
% Write your answer.

A simple test program is:

\begin{verbatim}
var primes[5];

procedure search(x, size)
begin
    if size = 0 then
        return;

    if primes[size - 1] = x then
        begin
            write size - 1;
            writeln
        end
    else
        call search(x, size - 1)
end;


begin
    primes[0] := 2;
    primes[1] := 3;
    primes[2] := 5;
    primes[3] := 7;
    primes[4] := 11;
    call search(7,5);
end.
\end{verbatim}

This test program exercises procedures using multiple parameters and recursive
procedure calls. The procedure takes a number to search for and an array size
and prints the array index of the number if it is in the array, otherwise it
does nothing.\\

My PL/0' compiler outputs:
\begin{verbatim}
; start compilation
; start execution
3
\end{verbatim}

%%%%%%%%%%%%%%%%%%%%%%%%%%%%%%%%%%%%%%%%%%%%%%%%%%%%%%%%%%%%%%%%%%%%%%%%%%%%%%%%%%%%%%%%%%%%%%%
\newpage
\section*{Question 6}
Introduce your own idea to your PL/0' compiler.

\ifreport
(Answer)\\
\fi
% Write your answer.

I introduced the \emph{for-do statement} into my PL/0' compiler. The
corresponding syntax diagram is as follows:

\begin{rail}
    statement: "for" "ident" ":=" expression ("to" | "down" "to") expression
               "do" statement
\end{rail}

The object code for \emph{for-do statement} is:\\

\begin{figure}[h]
\begin{tabular}{ll}
        & Object code for the variable initialisation \\
label1: & Object code for the condition \\
        & jpc label2 \\
        & Object code of {\it statement} \\
        & Object code for updating the "for variable" \\
        & jmp label1 \\
label2:
\end{tabular}
\end{figure}

I introduced two more instructions to the PL/0' virtual machine just to make
the code cleaner and more concise. They are:\\

\noindent\textbf{\Large uad}\\

\noindent\textbf{Overview}\\
\indent Increment by one the value of a variable. It is  a conjunction of
a lod, a lit, an opr,add and a sto instructions.\\

\noindent\textbf{Mnemonic}\\
\indent \texttt{uad,LEVEL,ADDR}\\

\noindent\textbf{Description}\\
\indent LEVEL is a nesting level in a source PL/0' program.
 ADDR is a relative address.\\

\noindent\textbf{Details}\\
\indent \texttt{stack[top++] = stack[display[i.u.addr.level] + i.u.addr.addr];\\
\indent stack[top] = 1;\\
\indent stack[top-1] += stack[top];\\
\indent stack[display[i.u.addr.level] + i.u.addr.addr] = stack[--top];}\\

\vspace{0.5cm}

\noindent\textbf{\Large usb}\\

\noindent\textbf{Overview}\\
\indent Decrement by one the value of a variable. It is a conjunction of
a lod, a lit, an opr,sub and a sto instructions.\\

\noindent\textbf{Mnemonic}\\
\indent \texttt{usb,LEVEL,ADDR}\\

\noindent\textbf{Description}\\
\indent LEVEL is a nesting level in a source PL/0' program.
 ADDR is a relative address.\\

\noindent\textbf{Details}\\
\indent \texttt{stack[top++] = stack[display[i.u.addr.level] +
    i.u.addr.addr];\\
\indent stack[top] = 1;\\
\indent stack[top-1] -= stack[top];\\
\indent stack[display[i.u.addr.level] + i.u.addr.addr] = stack[--top];}\\

To implement the \emph{for-do statement}, I did the following modifications:

\begin{itemize}
    \item In {\tt codegen.h} file, I added {\tt uad} and {\tt usb} instructions
        in the {\tt typedef enum codes OpCode}.
    \item In {\tt codegen.c} file, I implemented the new instructions's
        behaviour in the {\tt execute} function.
    \item In {\tt getSource.h} file, I added {\tt For}, {\tt Down} and {\tt To}
        keywords in the {\tt typedef  enum  keys KeyId}.
    \item In {\tt getSource.c} file, I added {\tt \{"for", For\}},
        {\tt \{"down", Down\}}  and {\tt \{"to", To\}} entries in the {\tt struct keyWd KeyWdT[]}.
    \item In {\tt compile.c} file, I modified the {\tt statement} function,
        including a For case in the main switch, implementing the compilation
        of the \emph{for-do statement}, supporting both incrementing and
        decrementing for loops. This implementation follows the syntax diagram
        shown above and the object code. It also uses the two new instructions
        {\tt uad} (for incrementing) and {\tt usb} (for decrementing) for
        updating the value of the for loop variable. And the {\tt For} token
        was added in the {\tt isStBeginKey} function as a {\tt statement}
        beginning token.
\end{itemize}

A simple test program is:

\clearpage

\begin{verbatim}
var i, a[10];

begin
    for i := 0 to 9 do
        a[i] := 2 * (i + 1);

    for i := 0 to 9 do begin
        write a[i];
        writeln
    end;

    writeln;

    for i := 9 down to 0 do begin
        write a[i];
        writeln
    end
end.
\end{verbatim}

This test program exercises \emph{for-do statement} in both incrementing and
decrementing versions, as well its combination with arrays.\\

My PL/0' compiler outputs:
\begin{verbatim}
; start compilation
; start execution
2 
4 
6 
8 
10 
12 
14 
16 
18 
20 

20 
18 
16 
14 
12 
10 
8 
6 
4 
2 
\end{verbatim}

\end{document}
